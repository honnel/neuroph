%% LaTeX-Beamer template for KIT design
%% by Erik Burger, Christian Hammer
%% title picture by Klaus Krogmann
%%
%% version 2.1
%%
%% mostly compatible to KIT corporate design v2.0
%% http://intranet.kit.edu/gestaltungsrichtlinien.php
%%
%% Problems, bugs and comments to
%% burger@kit.edu

\documentclass[18pt]{beamer}

%% SLIDE FORMAT

% use 'beamerthemekit' for standard 4:3 ratio
% for widescreen slides (16:9), use 'beamerthemekitwide'

\usepackage{templates/beamerthemekit}
\usepackage[utf8]{inputenc}
\usepackage[T1]{fontenc}
\usepackage{graphicx}
% \usepackage{templates/beamerthemekitwide}

%% TITLE PICTURE

% if a custom picture is to be used on the title page, copy it into the 'logos'
% directory, in the line below, replace 'mypicture' with the 
% filename (without extension) and uncomment the following line
% (picture proportions: 63 : 20 for standard, 169 : 40 for wide
% *.eps format if you use latex+dvips+ps2pdf, 
% *.jpg/*.png/*.pdf if you use pdflatex)

%\titleimage{mypicture}

%% TITLE LOGO

% for a custom logo on the front page, copy your file into the 'logos'
% directory, insert the filename in the line below and uncomment it

%\titlelogo{mylogo}

% (*.eps format if you use latex+dvips+ps2pdf,
% *.jpg/*.png/*.pdf if you use pdflatex)

%% TikZ INTEGRATION

% use these packages for PCM symbols and UML classes
% \usepackage{templates/tikzkit}
% \usepackage{templates/tikzuml}

% the presentation starts here

\title{Zwischenpräsentation der Java-Gruppe}
\subtitle{Neuronale Netze mit Neuroph}
\author{Markus Braun, Daniel Hammann, Dominik Messinger, Dominic Rausch}

\institute{Institut für Programmstrukturen und Datenorganisation (IPD), Lehrstuhl für Programmiersysteme}

% Bibliography

%\usepackage[citestyle=authoryear,bibstyle=numeric,hyperref,backend=biber]{biblatex}
%\addbibresource{templates/example.bib}
%\bibhang1em

\begin{document}
	\maketitle

	\begin{frame}[c]\frametitle{Neuronale Netze}
		\begin{center}
		\textit{Was machen wir überhaupt? Allgemeine (kurze) Einführung neuronale Netze}
		\end{center}
	\end{frame}

	\begin{frame}[c]\frametitle{Überblick}
		\begin{block}{Was bisher geschah}
		    \begin{itemize}
		    	\item Code Analyse
		    	\item Profiling (u.a. eigenes Evaluierungsframework)
		    	\item Verschiedene Parallelisierungsansätze durchprobiert
		    	\item Datensuche
		    \end{itemize}		    
		\end{block}
	\end{frame}
	
	\begin{frame}[c]\frametitle{Code Analyse \& Profiling}
		\begin{block}{}
			\begin{center}
			\textit{Sequenzdiagramme, Profiling-Zeiten, evntuell Aufbau von Neuroph anhand eines Klassendiagramms?, Evaluierungsframework erklären}
			\end{center}
		\end{block}
	\end {frame}
	
	\begin{frame}[c]\frametitle{Parallelisierungsansätze}
		\begin{block}{}
			MultiLayerPerceptron klonen
		\end{block}
		\begin{block}{}
			BatchParallelLearningRule
		\end{block}
	\end {frame}
	
	\begin{frame}[c]\frametitle{MultiLayerPerceptron klonen}
		\begin{block}{Aufbau}
			
		\end{block}
		\begin{block}{Probleme}
		
		\end{block}
	\end {frame}
	
		\begin{frame}[c]\frametitle{BatchParallelLearningRule}
		\begin{block}{Aufbau}
			
		\end{block}
		\begin{block}{Probleme}
		
		\end{block}
	\end {frame}
	
	\begin{frame}[c]\frametitle{Datensuche}
		\begin{block}{Erste Versuche}
		    \begin{itemize}
		    	\item StockExchange - Börsenvorhersage
		    	\item IrisScan Datensatz
		    \end{itemize}
		\end{block}
		\begin{block}{Teilchenkollision (Cern)}
		    \begin{itemize}
		    	\item Viele Daten
		    	\item Viele Neuronen :)
		    	\item Kompliziertes Daten Format
		    \end{itemize}
		\end{block}
	\end{frame}
	
	\begin{frame}[c]\frametitle{Whats next?}
		\begin{block}{Fahrplan}
		    \begin{itemize}
		    	\item Herr werden über die Cern-Daten
		    	\item Evaluierung der Ansätze
		    	\item ...
		    \end{itemize}
		\end{block}
	\end{frame}

\end{document}